% Filename: relatorio.tex
%
% This code is part of 'MT404 2012s2 - Proj04'
% 
% Description: Relatorio do Projeto No.3 de MT404 2012s2.
% 
% Created: 22.08.12 07:25:12 AM
% Last Change: 22.08.12 07:25:12 AM
% 
% Author:
% - Raniere Silva, <r.gaia.cs@gmail.com>
% 
% Copyright (c) 2012, Raniere Silva. All rights reserved.
% 
% Except otherwise noted, this work is licensed under the Creative
% Commons Attribution-ShareAlike 3.0 Unported License. To view a copy of
% this license, visit http://creativecommons.org/licenses/by-sa/3.0/ or
% send a letter to Creative Commons, 444 Castro Street, Suite 900,
% Mountain View, California, 94041, USA.
%
% This work is distributed in the hope that it will be useful, but
% WITHOUT ANY WARRANTY; without even the implied warranty of
% MERCHANTABILITY or FITNESS FOR A PARTICULAR PURPOSE.
%
\documentclass[12pt,a4paper]{article}
% Filename: packages.tex
% 
% This code is part of 'MT404 2012s2'
% 
% Description: This file corresponds to the packages used.
% 
% Created: 22.08.12 07:25:12 AM
% Last Change: 22.08.12 08:23:08 PM
% 
% Authors:
% - Raniere Silva (2012): initial version
% 
% Copyright (c) 2012 Raniere Silva <r.gaia.cs@gmail.com>
% 
% This work is licensed under the Creative Commons Attribution-ShareAlike 3.0 Unported License. To view a copy of this license, visit http://creativecommons.org/licenses/by-sa/3.0/ or send a letter to Creative Commons, 444 Castro Street, Suite 900, Mountain View, California, 94041, USA.
%
% This work is distributed in the hope that it will be useful, but WITHOUT ANY WARRANTY; without even the implied warranty of MERCHANTABILITY or FITNESS FOR A PARTICULAR PURPOSE.
%
\usepackage[utf8]{inputenc}
\usepackage[T1]{fontenc}

% Configura\c{c}\~{o}es regionais
\usepackage[top=3cm,left=2cm,right=2cm,bottom=3cm]{geometry}
\usepackage[brazil]{babel}
\usepackage{indentfirst}

% Pacotes matem\'{a}ticos
\usepackage{amsmath}
\usepackage{amsthm}
\usepackage{amsfonts}
\usepackage{amssymb}
% Customiza\c{c}\~{a}o do pacote amsmath
\newtheorem{defi}{Definição}
\newtheorem{prop}{Proposição}
\allowdisplaybreaks[4]

% Pacotes gr\'{a}ficos
\usepackage{graphicx}
\usepackage{tikz}

% Links
\usepackage{url}
\usepackage{hyperref}

% Algoritmos
\usepackage{algorithmic} % Pacote para algoritmos
\usepackage{algorithm} % Pacote para algoritmos
% Customiza\c{c}\~{a}o do pacote algorithm
\floatname{algorithm}{Algoritmo}
\algsetup{linenosize=\small}
\renewcommand{\algorithmicrequire}{\textbf{Entrada:}}
\renewcommand{\algorithmicensure}{\textbf{Saída:}}
\renewcommand{\algorithmicend}{\textbf{fim}}
\renewcommand{\algorithmicif}{\textbf{se}}
\renewcommand{\algorithmicthen}{\textbf{ent\~{a}o}}
\renewcommand{\algorithmicelse}{\textbf{caso contr\'{a}rio}}
\renewcommand{\algorithmicendif}{\algorithmicend}
\renewcommand{\algorithmicfor}{\textbf{para}}
\renewcommand{\algorithmicforall}{\textbf{para todo}}
\renewcommand{\algorithmicdo}{\textbf{fa\c{c}a}}
\renewcommand{\algorithmicendfor}{\algorithmicend}
\renewcommand{\algorithmicwhile}{\textbf{enquanto}}
\renewcommand{\algorithmicendwhile}{\algorithmicend}
\renewcommand{\algorithmicrepeat}{\textbf{repita}}
\renewcommand{\algorithmicuntil}{\textbf{at\'{e}}}
\renewcommand{\algorithmicreturn}{\textbf{retorne}}
\renewcommand{\algorithmiccomment}[1]{\hspace{2em}/* #1 */}

% C\'{o}digos
\usepackage{listings} % Pacote para c\'{o}digos
\usepackage{textcomp} % Para aspas simples
\renewcommand{\lstlistingname}{C\'{o}digo}
\lstdefinestyle{codes}{ %
language=Octave,                % the language of the code
frame=single,                   % frame around the code
basicstyle=\ttfamily\small,     % the size of the fonts that are used for the code
columns=flexible,               % Correct unexpected spaces in strings when copy-and-past
numbers=left,                   % where to put the line-numbers
numberstyle=\footnotesize,      % the size of the fonts that are used for the line-numbers
stepnumber=5,                   % the step between two line-numbers.
numbersep=5pt,                  % how far the line-numbers are from the code
% backgroundcolor=\color{white},  % choose the background color. You must add \usepackage{color}
showspaces=false,               % show spaces adding particular underscores
showstringspaces=false,         % underline spaces within strings
% showtabs=false,                 % show tabs within strings adding particular underscores
% frame=single,                   % adds a frame around the code
tabsize=4,                      % sets default tabsize to 2 spaces
captionpos=t,                   % sets the caption-position to bottom
breaklines=true,                % sets automatic line breaking
breakatwhitespace=false,        % sets if automatic breaks should only happen at whitespace
% caption={\texttt{\lstname}},    % show the filename of files included with \lstinputlisting;
% escapeinside={\%*}{*)},         % if you want to add a comment within your code
% morekeywords={#},               % if you want to add more keywords to the set
upquote=true,                   % Correct single quote
}
\lstdefinestyle{outputs}{ %
language=Octave,                % the language of the code
% frame=single,                   % frame around the code
basicstyle=\ttfamily\small,     % the size of the fonts that are used for the code
columns=flexible,               % Correct unexpected spaces in strings when copy-and-past
% numbers=left,                   % where to put the line-numbers
% numberstyle=\footnotesize,      % the size of the fonts that are used for the line-numbers
% stepnumber=5,                   % the step between two line-numbers.
% numbersep=5pt,                  % how far the line-numbers are from the code
% backgroundcolor=\color{white},  % choose the background color. You must add \usepackage{color}
showspaces=false,               % show spaces adding particular underscores
showstringspaces=false,         % underline spaces within strings
% showtabs=false,                 % show tabs within strings adding particular underscores
% frame=single,                   % adds a frame around the code
tabsize=4,                      % sets default tabsize to 2 spaces
% captionpos=t,                   % sets the caption-position to bottom
breaklines=true,                % sets automatic line breaking
breakatwhitespace=false,        % sets if automatic breaks should only happen at whitespace
% caption={\texttt{\lstname}},    % show the filename of files included with \lstinputlisting;
% escapeinside={\%*}{*)},         % if you want to add a comment within your code
% morekeywords={#}                % if you want to add more keywords to the set
upquote=true,                   % Correct single quote
}

% Novos comandos

\begin{document}
\title{Projeto No.4 de MT404}
\author{Raniere Silva \\ ra092767}
\maketitle
\begin{abstract}
    Este \'{e} o projeto no.4 da disciplina MT404 - M\'{e}todos Computacionais
    de \'{A}lgebra Linear.
    sistema linear.
\end{abstract}
\tableofcontents
\lstlistoflistings
\section*{Licen\c{c}a}
Salvo disposi\c{c}\~{a}o em contr\'{a}rio, este trabalho foi licenciado com uma
Licen\c{c}a Creative Commons Atribui\c{c}\~{a}o - CompartilhaIgual 3.0 N\~{a}o
Adaptada. Para ver uma c\'{o}pia desta licen\c{c}a, visite
http://creativecommons.org/licenses/by-sa/3.0/.
\begin{center}
    \includegraphics{../figuras/cc-by-sa.png}
\end{center}
\newpage
\section{Fatora\c{c}\~{a}o LU}
A maneira mais elementar de resolver um sistema de equa\c{c}\~{o}es \'{e} \'{e}
utilizando a elimina\c{c}\~{a}o de Gauss, i.e., em triangularizar o
sistema, e posteriormente utilizar substitui\c{c}\~{a}o para obter a
solu\c{c}\~{a}o do sistema.

A Fatora\c{c}\~{a}o LU \'{e} uma vers\~{a}o mais elaborada da elimina\c{c}\~{a}o
de Gauss e dado o sistema $A x = b$ encontrar matrizes $L$ e $U$, onde $L$ \'{e}
tringular inferior com diagonal unit\'{a}ria e $U$ triangular superior, tal que
$A = L U$. A solu\c{c}\~{a}o do sistema linear \'{e} obtida resolvendo dois
sistemas:
\begin{align*}
    L y &= b, \\
    U x &= y.
\end{align*}

\'{E} poss\'{i}vel provar que $A \in \mathbb{R}^{n \times n}$ possue
fatora\c{c}\~{a}o LU se os menores principais de $A$, de ordem $1, 2, \ldots, n
- 1$, s\~{a}o n\~{a}o nulos. Al\'{e}m disso, se a fatora\c{c}\~{a}o existir e
$A$ for n\~{a}o singular, ent\~{a}o a fatora\c{c}\~{a}o LU \'{e} \'{u}nica.

Sabe-se que para algumas matrizes n\~{a}o \'{e} poss\'{i}vel obter a
fatora\c{c}\~{a}o LU, e.g.,
\begin{align*}
    A = \begin{bmatrix}
        0 & 1 \\
        1 & 0
    \end{bmatrix}.
\end{align*}
Para algumas dessas matrizes, existe uma matriz $P$, $P$ \'{e} uma matriz de
permuta\c{c}\~{a}o, tal que existe a fatora\c{c}\~{a}o LU de $P A$.

No caso de matrizes esparsas, a fatora\c{c}\~{a}o LU pode causar o preenchimento
da matrizes $L$ e $U$, para evitar isso pode-se relaxar a permuta\c{c}\~{a}o
$P$.

\section{Experimentos Computacionais}
O GNU Octave possue implementado, no arquivo lu.oct, uma fun\c{c}\~{a}o para o c\'{a}lculo da
fatora\c{c}\~{a}o LU de uma matriz. Essa fun\c{c}\~{a}o recebe como
par\^{a}metro a matriz $A$ que deseja-se calcular os fatores $L$ e $U$ e,
opcionalmente, um escalar $\mathrm{THRES}$, $0 < \mathrm{THRES} \leq 1$, que
define uma toler\^{a}ncia na escolha do piv\^{o} a ser utilizado.

O objetivo do par\^{a}metro $\mathrm{THRES}$ \'{e} tentar minimizar o
preenchimento nas matrizes $L$ e $U$ de matrizes esparsas $A$.

Espera-se que:
\begin{enumerate}
    \item ao diminuir o valor de $\mathrm{THRES}$ a esparcidade de $L$ e $U$
        aumente devido a toler\^{a}ncia na escolha do piv\^{o}.
    \item o tempo para calcular a fatora\c{c}\~{a}o LU com $\mathrm{THRES} = 1$
        seja inferior ao demais e ao diminuir o valor de $\mathrm{THRES}$ o
        tempo tamb\'{e}m diminua pela redu\c{c}\~{a}o no n\'{u}mero de
        opera\c{c}\~{o}es.
    \item a precis\~{a}o da solu\c{c}\~{a}o seja superior quanto mais alto for o
        valor de $\mathrm{THRES}$.
\end{enumerate}

\section{C\'{o}digos}
A seguir encontra-se os c\'{o}digos desenvolvidos neste projeto. Todos os c'{o}digos
foram testados utilizando o GNU Octave em sua vers\~{a}o
3.2.4\footnote{Acredita-se que os c\'{o}digos sejam compat\'{i}eis com o MATLAB
embora n\~{a}o tenham sido testados nesse ambiente.} e encontram-se
dispon\'{i}veis em \url{https://github.com/r-gaia-cs/mt404-2012s2}.
\lstinputlisting[style=codes, caption={An\'{a}lise de sensibilidade},
label={code:mt404_p04}]{src/mt404_p04.m}
\end{document}
