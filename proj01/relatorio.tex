% Filename: relatorio.tex
%
% This code is part of 'MT404 2012s2 - Proj01'
% 
% Description: Relatorio do Projeto No.1 de MT404 2012s2.
% 
% Created: 22.08.12 07:25:12 AM
% Last Change: 22.08.12 07:25:12 AM
% 
% Author:
% - Raniere Silva, <r.gaia.cs@gmail.com>
% - Julio Cesar, <julioholanda7@gmail.com>
% 
% Copyright (c) 2012, Raniere Silva. All rights reserved.
% Copyright (c) 2012, Julio Cesar. All rights reserved.
% 
% Except otherwise noted, this work is licensed under the Creative Commons Attribution-ShareAlike 3.0 Unported License. To view a copy of this license, visit http://creativecommons.org/licenses/by-sa/3.0/ or send a letter to Creative Commons, 444 Castro Street, Suite 900, Mountain View, California, 94041, USA.
%
% This work is distributed in the hope that it will be useful, but WITHOUT ANY WARRANTY; without even the implied warranty of MERCHANTABILITY or FITNESS FOR A PARTICULAR PURPOSE.
%
\documentclass[12pt,a4paper]{article}
% Filename: packages.tex
% 
% This code is part of 'MT404 2012s2'
% 
% Description: This file corresponds to the packages used.
% 
% Created: 22.08.12 07:25:12 AM
% Last Change: 22.08.12 08:23:08 PM
% 
% Authors:
% - Raniere Silva (2012): initial version
% 
% Copyright (c) 2012 Raniere Silva <r.gaia.cs@gmail.com>
% 
% This work is licensed under the Creative Commons Attribution-ShareAlike 3.0 Unported License. To view a copy of this license, visit http://creativecommons.org/licenses/by-sa/3.0/ or send a letter to Creative Commons, 444 Castro Street, Suite 900, Mountain View, California, 94041, USA.
%
% This work is distributed in the hope that it will be useful, but WITHOUT ANY WARRANTY; without even the implied warranty of MERCHANTABILITY or FITNESS FOR A PARTICULAR PURPOSE.
%
\usepackage[utf8]{inputenc}
\usepackage[T1]{fontenc}

% Configura\c{c}\~{o}es regionais
\usepackage[top=3cm,left=2cm,right=2cm,bottom=3cm]{geometry}
\usepackage[brazil]{babel}
\usepackage{indentfirst}

% Pacotes matem\'{a}ticos
\usepackage{amsmath}
\usepackage{amsthm}
\usepackage{amsfonts}
\usepackage{amssymb}
% Customiza\c{c}\~{a}o do pacote amsmath
\newtheorem{defi}{Definição}
\newtheorem{prop}{Proposição}
\allowdisplaybreaks[4]

% Pacotes gr\'{a}ficos
\usepackage{graphicx}
\usepackage{tikz}

% Links
\usepackage{url}
\usepackage{hyperref}

% Algoritmos
\usepackage{algorithmic} % Pacote para algoritmos
\usepackage{algorithm} % Pacote para algoritmos
% Customiza\c{c}\~{a}o do pacote algorithm
\floatname{algorithm}{Algoritmo}
\algsetup{linenosize=\small}
\renewcommand{\algorithmicrequire}{\textbf{Entrada:}}
\renewcommand{\algorithmicensure}{\textbf{Saída:}}
\renewcommand{\algorithmicend}{\textbf{fim}}
\renewcommand{\algorithmicif}{\textbf{se}}
\renewcommand{\algorithmicthen}{\textbf{ent\~{a}o}}
\renewcommand{\algorithmicelse}{\textbf{caso contr\'{a}rio}}
\renewcommand{\algorithmicendif}{\algorithmicend}
\renewcommand{\algorithmicfor}{\textbf{para}}
\renewcommand{\algorithmicforall}{\textbf{para todo}}
\renewcommand{\algorithmicdo}{\textbf{fa\c{c}a}}
\renewcommand{\algorithmicendfor}{\algorithmicend}
\renewcommand{\algorithmicwhile}{\textbf{enquanto}}
\renewcommand{\algorithmicendwhile}{\algorithmicend}
\renewcommand{\algorithmicrepeat}{\textbf{repita}}
\renewcommand{\algorithmicuntil}{\textbf{at\'{e}}}
\renewcommand{\algorithmicreturn}{\textbf{retorne}}
\renewcommand{\algorithmiccomment}[1]{\hspace{2em}/* #1 */}

% C\'{o}digos
\usepackage{listings} % Pacote para c\'{o}digos
\usepackage{textcomp} % Para aspas simples
\renewcommand{\lstlistingname}{C\'{o}digo}
\lstdefinestyle{codes}{ %
language=Octave,                % the language of the code
frame=single,                   % frame around the code
basicstyle=\ttfamily\small,     % the size of the fonts that are used for the code
columns=flexible,               % Correct unexpected spaces in strings when copy-and-past
numbers=left,                   % where to put the line-numbers
numberstyle=\footnotesize,      % the size of the fonts that are used for the line-numbers
stepnumber=5,                   % the step between two line-numbers.
numbersep=5pt,                  % how far the line-numbers are from the code
% backgroundcolor=\color{white},  % choose the background color. You must add \usepackage{color}
showspaces=false,               % show spaces adding particular underscores
showstringspaces=false,         % underline spaces within strings
% showtabs=false,                 % show tabs within strings adding particular underscores
% frame=single,                   % adds a frame around the code
tabsize=4,                      % sets default tabsize to 2 spaces
captionpos=t,                   % sets the caption-position to bottom
breaklines=true,                % sets automatic line breaking
breakatwhitespace=false,        % sets if automatic breaks should only happen at whitespace
% caption={\texttt{\lstname}},    % show the filename of files included with \lstinputlisting;
% escapeinside={\%*}{*)},         % if you want to add a comment within your code
% morekeywords={#},               % if you want to add more keywords to the set
upquote=true,                   % Correct single quote
}
\lstdefinestyle{outputs}{ %
language=Octave,                % the language of the code
% frame=single,                   % frame around the code
basicstyle=\ttfamily\small,     % the size of the fonts that are used for the code
columns=flexible,               % Correct unexpected spaces in strings when copy-and-past
% numbers=left,                   % where to put the line-numbers
% numberstyle=\footnotesize,      % the size of the fonts that are used for the line-numbers
% stepnumber=5,                   % the step between two line-numbers.
% numbersep=5pt,                  % how far the line-numbers are from the code
% backgroundcolor=\color{white},  % choose the background color. You must add \usepackage{color}
showspaces=false,               % show spaces adding particular underscores
showstringspaces=false,         % underline spaces within strings
% showtabs=false,                 % show tabs within strings adding particular underscores
% frame=single,                   % adds a frame around the code
tabsize=4,                      % sets default tabsize to 2 spaces
% captionpos=t,                   % sets the caption-position to bottom
breaklines=true,                % sets automatic line breaking
breakatwhitespace=false,        % sets if automatic breaks should only happen at whitespace
% caption={\texttt{\lstname}},    % show the filename of files included with \lstinputlisting;
% escapeinside={\%*}{*)},         % if you want to add a comment within your code
% morekeywords={#}                % if you want to add more keywords to the set
upquote=true,                   % Correct single quote
}

% Novos comandos

\begin{document}
\title{Projeto No.1 de MT404}
\author{Raniere Silva \\ ra092767  \and Julio Cesar \\ ra984581}
\maketitle
\begin{abstract}
    Este \'{e} o projeto no.1 da disciplina MT404 - M\'{e}todos Computacionais de \'{A}lgebra Linear. Neste projeto \'{e} abordado algumas opera\c{c}\~{o}es com vetores e matrizes. Duas pequenas demonstra\c{c}\~{o}es s\~{a}o apresentadas e os c\'{o}digos produzidos encontram-se no final.
\end{abstract}
\tableofcontents
\lstlistoflistings
\section*{Licen\c{c}a}
Salvo disposi\c{c}\~{a}o em contr\'{a}rio, este trabalho foi licenciado com uma Licen\c{c}a Creative Commons Atribui\c{c}\~{a}o - CompartilhaIgual 3.0 N\~{a}o Adaptada License. Para ver uma c\'{o}pia desta licen\c{c}a, visite http://creativecommons.org/licenses/by-sa/3.0/.
\begin{center}
    \includegraphics{../figuras/cc-by-sa.png}
\end{center}
\newpage
\section{Pot\^{e}ncia de Matrizes da forma $xy^T$}
\begin{prop}
    Dados $x, y \in \mathbb{R}^n$ então é válido
    \begin{align}
        (x y^T)^k = (x^T y)^{k - 1} (x y^T), \label{eq:pot_matriz}
    \end{align}
para $k = 1, 2, \ldots$.
\end{prop}
\begin{proof}
    Vamos provar utilizando indução finita sobre $k$.
    
    Para $k = 1$ temos que
    \begin{align*}
        & (x y^T)^1 = (x y^T), \\
        & (x^T y)^{1 - 1} (x y^T) = (x^T y)^0 (x y^T) = (x y^T)
    \end{align*}
    e, portanto \eqref{eq:pot_matriz} é válido.
    
    Assumindo que \eqref{eq:pot_matriz} é válido para $k = n$, vamos mostrar que também é válido para $k = n + 1$:
    \begin{align*}
        (x y^T)^{n + 1} &= (x y^T)^n (x y^T)  \\
        &= (x^T y)^{n - 1} (x y^T) (x y^T) && \text{por \eqref{eq:pot_matriz} para $k = n$}\\
        &= (x^T y)^{n - 1} x (y^T x) y^T \\
        &= (x^T y)^{n - 1} (y^T x) (x y^T) \\
        &= (x^T y)^{n - 1} (x^T y) (x y^T) \\
        &= (x^T y)^{n} (x y^T).
    \end{align*}
    Pelo resultado acima, $(x y^T)^{n + 1} = (x^T y)^{n} (x y^T)$ e assim concluimos nossa demonstração.
\end{proof}

O Código~\ref{code:mt404_p01q01b} corresponde a implementação de de $C = (x y^T)^k$ utilizando o resultado da proposição anterior. A seguir, apresentamos parte da saída obtida ao executarmos o Código~\ref{code:mt404_p01q01b} em um de nossos testes.
\lstinputlisting[style=outputs, nolol=true]{src/mt404_p01q01b.out}

\section{Produto de Matriz por vetor}
O Código~\ref{code:mt404_p01q02a} corresponde a implementação do produto de matriz por vetor na sua versão por coluna.
\lstinputlisting[style=outputs, nolol=true]{src/mt404_p01q02a.out}

O Código~\ref{code:mt404_p01q02b} corresponde a versão modificada do Código~\ref{code:mt404_p01q02a} para o produto de uma matriz de banda $2r + 1$ por um vetor de maneira eficiente.
\lstinputlisting[style=outputs, nolol=true]{src/mt404_p01q02b.out}

\section{Quadrado de Matrizes Triangulares}
\begin{prop}
    Seja $A \in \mathbb{R}^{n \times n}$ uma matriz triangular superior. Então $A^2 = A A$ também é uma matriz triangular superior.
\end{prop}
\begin{proof}
    Seja $A = (a_{ij})$, $i, j \in \{1, \ldots, n\}$, uma matriz triangular superior, i.e., $a_{ij} = 0$ se $i > j$.
    
    Como o produto de duas matrizes $B, C \in \mathbb{R}^{n \times n}$ é dado por
    \begin{align*}
        (B C)_{ij} &= \sum_{k = 1}^n b_{ik} c_{kj},
    \end{align*}
    para $i,j \in \{1, \ldots, n\}$, então $A^2$ observa-se que
    \begin{align*}
        (A^2)_{ij} &= \sum_{k = 1}^n a_{ik} a_{kj}.
    \end{align*}
    Mas $a_{ik} = 0$ se $i > k$ e $a_{jk} = 0$ se $k > j$, então por transitividade $i > k > j$ que implica em $(A^2)_{ij} = 0$ para $i > j$ e portanto $A^2$ também é uma matriz triangular superior.
\end{proof}

O Código~\ref{code:mt404_p01q03b} corresponde a implementação do quadrado de uma matriz triangular superior em que o resultado é armazenado na matriz de entrada. A seguir, apresentamos parte da saída obtida ao executarmos o Código~\ref{code:mt404_p01q03b} em um de nossos testes.
%\lstinputlisting[style=outputs, nolol=true]{src/mt404_p01q03b.out}

\section{C\'{o}digos}
\lstinputlisting[style=codes, caption={Quest\~{a}o 01, item b}, label={code:mt404_p01q01b}]{src/mt404_p01q01b.m}
\lstinputlisting[style=codes, caption={Quest\~{a}o 02, item a}, label={code:mt404_p01q02a}]{src/mt404_p01q02a.m}
\lstinputlisting[style=codes, caption={Quest\~{a}o 02, item b}, label={code:mt404_p01q02b}]{src/mt404_p01q02b.m}
\lstinputlisting[style=codes, caption={Quest\~{a}o 03, item b}, label={code:mt404_p01q03b}]{src/mt404_p01q03b.m}
\end{document}


