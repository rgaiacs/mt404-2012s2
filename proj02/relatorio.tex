% Filename: relatorio.tex
%
% This code is part of 'MT404 2012s2 - Proj02'
% 
% Description: Relatorio do Projeto No.2 de MT404 2012s2.
% 
% Created: 22.08.12 07:25:12 AM
% Last Change: 22.08.12 07:25:12 AM
% 
% Author:
% - Raniere Silva, <r.gaia.cs@gmail.com>
% 
% Copyright (c) 2012, Raniere Silva. All rights reserved.
% 
% Except otherwise noted, this work is licensed under the Creative Commons Attribution-ShareAlike 3.0 Unported License. To view a copy of this license, visit http://creativecommons.org/licenses/by-sa/3.0/ or send a letter to Creative Commons, 444 Castro Street, Suite 900, Mountain View, California, 94041, USA.
%
% This work is distributed in the hope that it will be useful, but WITHOUT ANY WARRANTY; without even the implied warranty of MERCHANTABILITY or FITNESS FOR A PARTICULAR PURPOSE.
%
\documentclass[12pt,a4paper]{article}
% Filename: packages.tex
% 
% This code is part of 'MT404 2012s2'
% 
% Description: This file corresponds to the packages used.
% 
% Created: 22.08.12 07:25:12 AM
% Last Change: 22.08.12 08:23:08 PM
% 
% Authors:
% - Raniere Silva (2012): initial version
% 
% Copyright (c) 2012 Raniere Silva <r.gaia.cs@gmail.com>
% 
% This work is licensed under the Creative Commons Attribution-ShareAlike 3.0 Unported License. To view a copy of this license, visit http://creativecommons.org/licenses/by-sa/3.0/ or send a letter to Creative Commons, 444 Castro Street, Suite 900, Mountain View, California, 94041, USA.
%
% This work is distributed in the hope that it will be useful, but WITHOUT ANY WARRANTY; without even the implied warranty of MERCHANTABILITY or FITNESS FOR A PARTICULAR PURPOSE.
%
\usepackage[utf8]{inputenc}
\usepackage[T1]{fontenc}

% Configura\c{c}\~{o}es regionais
\usepackage[top=3cm,left=2cm,right=2cm,bottom=3cm]{geometry}
\usepackage[brazil]{babel}
\usepackage{indentfirst}

% Pacotes matem\'{a}ticos
\usepackage{amsmath}
\usepackage{amsthm}
\usepackage{amsfonts}
\usepackage{amssymb}
% Customiza\c{c}\~{a}o do pacote amsmath
\newtheorem{defi}{Definição}
\newtheorem{prop}{Proposição}
\allowdisplaybreaks[4]

% Pacotes gr\'{a}ficos
\usepackage{graphicx}
\usepackage{tikz}

% Links
\usepackage{url}
\usepackage{hyperref}

% Algoritmos
\usepackage{algorithmic} % Pacote para algoritmos
\usepackage{algorithm} % Pacote para algoritmos
% Customiza\c{c}\~{a}o do pacote algorithm
\floatname{algorithm}{Algoritmo}
\algsetup{linenosize=\small}
\renewcommand{\algorithmicrequire}{\textbf{Entrada:}}
\renewcommand{\algorithmicensure}{\textbf{Saída:}}
\renewcommand{\algorithmicend}{\textbf{fim}}
\renewcommand{\algorithmicif}{\textbf{se}}
\renewcommand{\algorithmicthen}{\textbf{ent\~{a}o}}
\renewcommand{\algorithmicelse}{\textbf{caso contr\'{a}rio}}
\renewcommand{\algorithmicendif}{\algorithmicend}
\renewcommand{\algorithmicfor}{\textbf{para}}
\renewcommand{\algorithmicforall}{\textbf{para todo}}
\renewcommand{\algorithmicdo}{\textbf{fa\c{c}a}}
\renewcommand{\algorithmicendfor}{\algorithmicend}
\renewcommand{\algorithmicwhile}{\textbf{enquanto}}
\renewcommand{\algorithmicendwhile}{\algorithmicend}
\renewcommand{\algorithmicrepeat}{\textbf{repita}}
\renewcommand{\algorithmicuntil}{\textbf{at\'{e}}}
\renewcommand{\algorithmicreturn}{\textbf{retorne}}
\renewcommand{\algorithmiccomment}[1]{\hspace{2em}/* #1 */}

% C\'{o}digos
\usepackage{listings} % Pacote para c\'{o}digos
\usepackage{textcomp} % Para aspas simples
\renewcommand{\lstlistingname}{C\'{o}digo}
\lstdefinestyle{codes}{ %
language=Octave,                % the language of the code
frame=single,                   % frame around the code
basicstyle=\ttfamily\small,     % the size of the fonts that are used for the code
columns=flexible,               % Correct unexpected spaces in strings when copy-and-past
numbers=left,                   % where to put the line-numbers
numberstyle=\footnotesize,      % the size of the fonts that are used for the line-numbers
stepnumber=5,                   % the step between two line-numbers.
numbersep=5pt,                  % how far the line-numbers are from the code
% backgroundcolor=\color{white},  % choose the background color. You must add \usepackage{color}
showspaces=false,               % show spaces adding particular underscores
showstringspaces=false,         % underline spaces within strings
% showtabs=false,                 % show tabs within strings adding particular underscores
% frame=single,                   % adds a frame around the code
tabsize=4,                      % sets default tabsize to 2 spaces
captionpos=t,                   % sets the caption-position to bottom
breaklines=true,                % sets automatic line breaking
breakatwhitespace=false,        % sets if automatic breaks should only happen at whitespace
% caption={\texttt{\lstname}},    % show the filename of files included with \lstinputlisting;
% escapeinside={\%*}{*)},         % if you want to add a comment within your code
% morekeywords={#},               % if you want to add more keywords to the set
upquote=true,                   % Correct single quote
}
\lstdefinestyle{outputs}{ %
language=Octave,                % the language of the code
% frame=single,                   % frame around the code
basicstyle=\ttfamily\small,     % the size of the fonts that are used for the code
columns=flexible,               % Correct unexpected spaces in strings when copy-and-past
% numbers=left,                   % where to put the line-numbers
% numberstyle=\footnotesize,      % the size of the fonts that are used for the line-numbers
% stepnumber=5,                   % the step between two line-numbers.
% numbersep=5pt,                  % how far the line-numbers are from the code
% backgroundcolor=\color{white},  % choose the background color. You must add \usepackage{color}
showspaces=false,               % show spaces adding particular underscores
showstringspaces=false,         % underline spaces within strings
% showtabs=false,                 % show tabs within strings adding particular underscores
% frame=single,                   % adds a frame around the code
tabsize=4,                      % sets default tabsize to 2 spaces
% captionpos=t,                   % sets the caption-position to bottom
breaklines=true,                % sets automatic line breaking
breakatwhitespace=false,        % sets if automatic breaks should only happen at whitespace
% caption={\texttt{\lstname}},    % show the filename of files included with \lstinputlisting;
% escapeinside={\%*}{*)},         % if you want to add a comment within your code
% morekeywords={#}                % if you want to add more keywords to the set
upquote=true,                   % Correct single quote
}

% Novos comandos

\begin{document}
\title{Projeto No.2 de MT404}
\author{Raniere Silva \\ ra092767}
\maketitle
\begin{abstract}
    Este \'{e} o projeto no.2 da disciplina MT404 - M\'{e}todos Computacionais de \'{A}lgebra Linear. Neste projeto abordamos algumas formas de armazenar matrizes esparsas, mais especificamente matrizes de banda. Implementamos o produto de uma matriz por um vetor para algumas das formas de armazenamento abordados e comparamos o tempo utilizado para calcular o produto.
\end{abstract}
\tableofcontents
\lstlistoflistings
\section*{Licen\c{c}a}
Salvo disposi\c{c}\~{a}o em contr\'{a}rio, este trabalho foi licenciado com uma Licen\c{c}a Creative Commons Atribui\c{c}\~{a}o - CompartilhaIgual 3.0 N\~{a}o Adaptada. Para ver uma c\'{o}pia desta licen\c{c}a, visite http://creativecommons.org/licenses/by-sa/3.0/.
\begin{center}
    \includegraphics{../figuras/cc-by-sa.png}
\end{center}
\newpage
\section{Armazenamento de Matrizes}
Considere $A \in \mathbb{R}^{m \times n}$ uma matriz de banda com banda superior com amplitude $q$ e banda inferior com amplitude $p$, i.e., $A$ \'{e} da forma
\begin{align*}
    \begin{bmatrix}
        a_{1,1} & a_{1,2} & \ldots & a_{1,q} & 0 & 0 & \ldots & 0 \\
        a_{2,1} & a_{2,2} & \ldots & a_{2,q} & a_{2,q+1} & 0 & \ldots & 0 \\
        \vdots & \vdots & \ddots & \vdots & \vdots & \vdots & \ddots & \vdots \\
        a_{p,1} & a_{p,2} & \ldots & a_{p,q} & a_{p,q+1} & a_{p,q+2} & \ldots & 0 \\
        0 & a_{p+1,2} & \ldots & a_{p+1,q} & a_{p+1,q+1} & a_{p+1,q+2} & \ldots & 0 \\
        \vdots & \vdots & \ddots & \vdots & \vdots & \vdots & \ddots & \vdots \\
        0 & 0 & \ldots & 0 & 0 & 0 & \ldots & a_{mn} \\
    \end{bmatrix}.
\end{align*}

Existe ao menos tr\^{e}s estruturas de dados que podem ser utilizadas para armazenar a matriz $A$ em um computador. A primeira delas n\~{a}o aproveita a estrutura especial da matriz, i.e., a matriz \'{e} armazenada como um vetor de vetores (ver a Figura~\ref{fig:est_dados_matriz} para o caso em que $m = n = 5$ e $q = p = 1$).
\begin{figure}[!htb]
    \centering
    \begin{tikzpicture}
        \node[below right] (A) at (0,0) {$A$};
        \node[below right] (B) at (2,0) {
            \begin{tabular}{|c|c|c|c|c|}
                \hline
                $a_{11}$ & $a_{12}$ & $0$ & $0$ & $0$ \\ \hline
                $a_{21}$ & $a_{22}$ & $a_{23}$ & $0$ & $0$ \\ \hline
                $0$ & $a_{32}$ & $a_{33}$ & $a_{34}$ & $0$ \\ \hline
                $0$ & $0$ & $a_{43}$ & $a_{44}$ & $a_{45}$ \\ \hline
                $0$ & $0$ & $0$ & $a_{54}$ & $a_{55}$ \\ \hline
            \end{tabular}};
        \draw[->] (.5,-.25) -- (2,-.25);
    \end{tikzpicture}
    \caption{Estrutura de dados para matriz tridiagonal sem aproveitamento da estrutura especial da matriz.}
    \label{fig:est_dados_matriz}
\end{figure}

A segunda maneira, muito utilizada para matrizes esparsas gen\'{e}ricas, consiste em tr\^{e}s vetores, $i$, $j$ e $v$, tal que $v_k = A_{i_k, j_k}$ para todo elemento não nulo da matriz $A$ (ver a Figura~\ref{fig:est_dados_nao_nulos} para o caso em que $m = n = 5$ e $q = p = 1$).
\begin{figure}[!htb]
    \centering
    \begin{tikzpicture}
        \node[below right] (i) at (0,0) {$i$};
        \node[below right] (vi) at (2,0) {
            \begin{tabular}{|p{.75cm}|p{.75cm}|p{.75cm}|p{.75cm}|p{.75cm}|p{.75cm}|p{.75cm}|p{.75cm}|}
                \hline
                $1$ & $1$ & $2$ & $2$ & $2$ & \ldots & $5$ \\ \hline
            \end{tabular}};
        \draw[->] (.5,-.25) -- (2,-.25);
        \node[below right] (j) at (0,-1) {$j$};
        \node[below right] (vj) at (2,-1) {
            \begin{tabular}{|p{.75cm}|p{.75cm}|p{.75cm}|p{.75cm}|p{.75cm}|p{.75cm}|p{.75cm}|p{.75cm}|}
                \hline
                $1$ & $2$ & $1$ & $2$ & $3$ & \ldots & $5$ \\ \hline
            \end{tabular}};
            \draw[->] (.5,-1.25) -- (2,-1.25);
        \node[below right] (a) at (0,-2) {$v$};
        \node[below right] (va) at (2,-2) {
            \begin{tabular}{|p{.75cm}|p{.75cm}|p{.75cm}|p{.75cm}|p{.75cm}|p{.75cm}|p{.75cm}|p{.75cm}|}
                \hline
                $a_{11}$ & $a_{12}$ & $a_{21}$ & $a_{22}$ & $a_{23}$ & \ldots & $a_{55}$ \\ \hline
            \end{tabular}};
            \draw[->] (.5,-2.25) -- (2,-2.25);
    \end{tikzpicture}
    \caption{Estrutura de dados para matriz tridiagonal armazenando apenas elementos n\~{a}o nulos.}
    \label{fig:est_dados_nao_nulos}
\end{figure}

J\'{a} a terceira e \'{u}ltima maneira, utilizada apenas para matrizes de banda, consiste em armazenar as diagonais n\~{a}o nulas da matriz $A$ como colunas de uma outra matriz que é armazenada sem o aproveitamento de sua estrutura (ver a Figura~\ref{fig:est_dados_diag2col} para o caso em que $m = n = 5$ e $q = p = 1$).
\begin{figure}[!htb]
    \centering
    \begin{tikzpicture}
        \node[below right] (A) at (0,0) {$A$};
        \node[below right] (B) at (2,0) {
            \begin{tabular}{|c|c|c|c|c|}
                \hline
                $a_{21}$ & $a_{11}$ & $0$ \\ \hline
                $a_{32}$ & $a_{22}$ & $a_{12}$ \\ \hline
                $a_{43}$ & $a_{33}$ & $a_{23}$ \\ \hline
                $a_{54}$ & $a_{44}$ & $a_{34}$ \\ \hline
                $0$ & $a_{55}$ & $a_{45}$ \\ \hline
            \end{tabular}};
        \draw[->] (.5,-.25) -- (2,-.25);
    \end{tikzpicture}
    \caption{Estrutura de dados para matriz tridiagonal armazenando as diagonais como colunas.}
    \label{fig:est_dados_diag2col}
\end{figure}

\section{Teste Computacional}
Implementou-se, ver C\'{o}digo~\ref{code:mt404_p02}, a multiplica\c{c}\~{a}o de uma matriz $A \in \mathbb{R}^{m \times n}$ de banda (banda superior com amplitude $q$ e banda inferior com amplitude $p$) por um vetor $x \in \mathbb{R}^n$ para as tr\^{e}s maneira de armazenar a matriz $A$ apresentadas na se\c{c}\~{a}o anterior.

Testou-se a implementa\c{c}\~{a}o com matrizes pentadiagonais, $p = q = 2$, geradas aleatoriamente, de dimens\~{o}es $m = n = 500, 1000, 2000, 5000, 6000$, e a sa\'{i}da obtida para um dos testes foi:
\lstinputlisting[style=outputs, nolol=true]{src/mt404_p02.out}

A sa\'{i}da apresentada acima corresponde aos tempos utilizados para calcular o produto da matriz $A$ pelo vetor $x$, ver a Tabela~\ref{tab:tempo_prod}, para as diferentes forma de armazenamento e as diferentes dimensões da matriz $A$.
\begin{table}[!htb]
    \centering
    \caption{Tempo do produto de matriz de banda pelo tipo de armazenamento.}
    \label{tab:tempo_prod}
    \begin{tabular}{|c|c|c|c|c|c|}
        \hline
        & \multicolumn{5}{|c|}{Valores de $m = n$ da matriz $A \in \mathbb{R}^{m \times n}$} \\ \cline{2-6}
        Armazenamento & $500$ & $1000$ & $2000$ & $5000$ & $6000$ \\ \hline
        Figura~\ref{fig:est_dados_matriz} & 2.0387e--02 & 3.0580e--03 & 1.2127e--02 & 7.2978e--02 & 1.4011e--01 \\ \hline
        Figura~\ref{fig:est_dados_nao_nulos} & 1.0107e--02 & 1.3899e--04 & 2.6698e--04 & 5.6897e--04 & 1.5927e--02 \\ \hline
        Figura~\ref{fig:est_dados_diag2col} & 1.3680e--01 & 2.7278e--01 & 5.5286e--01 & 1.3628e+00 & 1.6472e+00 \\ \hline
    \end{tabular}
\end{table}

Observando a Tabela~\ref{tab:tempo_prod} notamos que os menores tempos foram obtidos ao armazenar apenas os elementos não nulos de $A$ utilizando três vetores. Já os maiores tempos foram obtidos ao armazenar as diagonais de $A$ como colunas de uma outra matriz.

Os tempos presentes na Tabela~\ref{tab:tempo_prod} podem ser justificados pelos seguintes fatos:
\begin{enumerate}
    \item Para o produto sem aproveitamento da estrutura esparsa e armazenando apenas os elementos não nulos de $A$ utilizando três vetores foi utilizada a função interna de multiplicação que é escrita em C++ e por isso o tempo utilizado por ambas as formas de armazenamento é inferior ao armazenar as diagonais de $A$ como colunas de uma outra matriz (o produto para essa forma de armazenamento foi implementado utilizando as funções do GNU Octave),
    \item Para o produto sem aproveitamento da estrutura esparsa é realizado várias operações com elementos nulos que não ocorrem ao armazenar apenas os elementos não nulos de $A$ utilizando três vetores e por isso o armazenamento utilizando três vetores para os elementos não nulos de $A$ foi mais rápido.
\end{enumerate}

\section{C\'{o}digos}
A seguir encontra-se os códigos desenvolvidos neste projeto. Todos os códigos foram testados utilizando o GNU Octave em sua versão 3.2.4\footnote{Acredita-se que os códigos sejam compatíveis com o MATLAB embora não tenham sido testados nesse ambiente.} e encontram-se disponíveis em \url{https://github.com/r-gaia-cs/mt404-2012s2}.
\lstinputlisting[style=codes, caption={Produto de matrizes}, label={code:mt404_p02}]{src/mt404_p02.m}
\end{document}
