% Filename: relatorio.tex
%
% This code is part of 'MT404 2012s2 - Proj04'
% 
% Description: Relatorio do Projeto No.5 de MT404 2012s2.
% 
% Created: 22.08.12 07:25:12 AM
% Last Change: 22.08.12 07:25:12 AM
% 
% Author:
% - Raniere Silva, <r.gaia.cs@gmail.com>
% 
% Copyright (c) 2012, Raniere Silva. All rights reserved.
% 
% Except otherwise noted, this work is licensed under the Creative
% Commons Attribution-ShareAlike 3.0 Unported License. To view a copy of
% this license, visit http://creativecommons.org/licenses/by-sa/3.0/ or
% send a letter to Creative Commons, 444 Castro Street, Suite 900,
% Mountain View, California, 94041, USA.
%
% This work is distributed in the hope that it will be useful, but
% WITHOUT ANY WARRANTY; without even the implied warranty of
% MERCHANTABILITY or FITNESS FOR A PARTICULAR PURPOSE.
%
\documentclass[12pt,a4paper]{article}
% Filename: packages.tex
% 
% This code is part of 'MT404 2012s2'
% 
% Description: This file corresponds to the packages used.
% 
% Created: 22.08.12 07:25:12 AM
% Last Change: 22.08.12 08:23:08 PM
% 
% Authors:
% - Raniere Silva (2012): initial version
% 
% Copyright (c) 2012 Raniere Silva <r.gaia.cs@gmail.com>
% 
% This work is licensed under the Creative Commons Attribution-ShareAlike 3.0 Unported License. To view a copy of this license, visit http://creativecommons.org/licenses/by-sa/3.0/ or send a letter to Creative Commons, 444 Castro Street, Suite 900, Mountain View, California, 94041, USA.
%
% This work is distributed in the hope that it will be useful, but WITHOUT ANY WARRANTY; without even the implied warranty of MERCHANTABILITY or FITNESS FOR A PARTICULAR PURPOSE.
%
\usepackage[utf8]{inputenc}
\usepackage[T1]{fontenc}

% Configura\c{c}\~{o}es regionais
\usepackage[top=3cm,left=2cm,right=2cm,bottom=3cm]{geometry}
\usepackage[brazil]{babel}
\usepackage{indentfirst}

% Pacotes matem\'{a}ticos
\usepackage{amsmath}
\usepackage{amsthm}
\usepackage{amsfonts}
\usepackage{amssymb}
% Customiza\c{c}\~{a}o do pacote amsmath
\newtheorem{defi}{Definição}
\newtheorem{prop}{Proposição}
\allowdisplaybreaks[4]

% Pacotes gr\'{a}ficos
\usepackage{graphicx}
\usepackage{tikz}

% Links
\usepackage{url}
\usepackage{hyperref}

% Algoritmos
\usepackage{algorithmic} % Pacote para algoritmos
\usepackage{algorithm} % Pacote para algoritmos
% Customiza\c{c}\~{a}o do pacote algorithm
\floatname{algorithm}{Algoritmo}
\algsetup{linenosize=\small}
\renewcommand{\algorithmicrequire}{\textbf{Entrada:}}
\renewcommand{\algorithmicensure}{\textbf{Saída:}}
\renewcommand{\algorithmicend}{\textbf{fim}}
\renewcommand{\algorithmicif}{\textbf{se}}
\renewcommand{\algorithmicthen}{\textbf{ent\~{a}o}}
\renewcommand{\algorithmicelse}{\textbf{caso contr\'{a}rio}}
\renewcommand{\algorithmicendif}{\algorithmicend}
\renewcommand{\algorithmicfor}{\textbf{para}}
\renewcommand{\algorithmicforall}{\textbf{para todo}}
\renewcommand{\algorithmicdo}{\textbf{fa\c{c}a}}
\renewcommand{\algorithmicendfor}{\algorithmicend}
\renewcommand{\algorithmicwhile}{\textbf{enquanto}}
\renewcommand{\algorithmicendwhile}{\algorithmicend}
\renewcommand{\algorithmicrepeat}{\textbf{repita}}
\renewcommand{\algorithmicuntil}{\textbf{at\'{e}}}
\renewcommand{\algorithmicreturn}{\textbf{retorne}}
\renewcommand{\algorithmiccomment}[1]{\hspace{2em}/* #1 */}

% C\'{o}digos
\usepackage{listings} % Pacote para c\'{o}digos
\usepackage{textcomp} % Para aspas simples
\renewcommand{\lstlistingname}{C\'{o}digo}
\lstdefinestyle{codes}{ %
language=Octave,                % the language of the code
frame=single,                   % frame around the code
basicstyle=\ttfamily\small,     % the size of the fonts that are used for the code
columns=flexible,               % Correct unexpected spaces in strings when copy-and-past
numbers=left,                   % where to put the line-numbers
numberstyle=\footnotesize,      % the size of the fonts that are used for the line-numbers
stepnumber=5,                   % the step between two line-numbers.
numbersep=5pt,                  % how far the line-numbers are from the code
% backgroundcolor=\color{white},  % choose the background color. You must add \usepackage{color}
showspaces=false,               % show spaces adding particular underscores
showstringspaces=false,         % underline spaces within strings
% showtabs=false,                 % show tabs within strings adding particular underscores
% frame=single,                   % adds a frame around the code
tabsize=4,                      % sets default tabsize to 2 spaces
captionpos=t,                   % sets the caption-position to bottom
breaklines=true,                % sets automatic line breaking
breakatwhitespace=false,        % sets if automatic breaks should only happen at whitespace
% caption={\texttt{\lstname}},    % show the filename of files included with \lstinputlisting;
% escapeinside={\%*}{*)},         % if you want to add a comment within your code
% morekeywords={#},               % if you want to add more keywords to the set
upquote=true,                   % Correct single quote
}
\lstdefinestyle{outputs}{ %
language=Octave,                % the language of the code
% frame=single,                   % frame around the code
basicstyle=\ttfamily\small,     % the size of the fonts that are used for the code
columns=flexible,               % Correct unexpected spaces in strings when copy-and-past
% numbers=left,                   % where to put the line-numbers
% numberstyle=\footnotesize,      % the size of the fonts that are used for the line-numbers
% stepnumber=5,                   % the step between two line-numbers.
% numbersep=5pt,                  % how far the line-numbers are from the code
% backgroundcolor=\color{white},  % choose the background color. You must add \usepackage{color}
showspaces=false,               % show spaces adding particular underscores
showstringspaces=false,         % underline spaces within strings
% showtabs=false,                 % show tabs within strings adding particular underscores
% frame=single,                   % adds a frame around the code
tabsize=4,                      % sets default tabsize to 2 spaces
% captionpos=t,                   % sets the caption-position to bottom
breaklines=true,                % sets automatic line breaking
breakatwhitespace=false,        % sets if automatic breaks should only happen at whitespace
% caption={\texttt{\lstname}},    % show the filename of files included with \lstinputlisting;
% escapeinside={\%*}{*)},         % if you want to add a comment within your code
% morekeywords={#}                % if you want to add more keywords to the set
upquote=true,                   % Correct single quote
}

% Novos comandos

\usepackage{csvsimple}
\begin{document}
\title{Projeto No.5 de MT404}
\author{Raniere Silva \\ ra092767}
\maketitle
\begin{abstract}
    Este \'{e} o projeto no.5 da disciplina MT404 - M\'{e}todos Computacionais
    de \'{A}lgebra Linear. Neste projeto investigou-se a Fatoração de Cholesky
    que existe apenas para matrizes simétrica definida positiva e o
    comportamento de matrizes simétrica definida positiva mal-condicionada
    para as quais pode ocorrer de numericamente não existir a Fatoração de
    Cholesky.
\end{abstract}
\tableofcontents
\lstlistoflistings
\section*{Licen\c{c}a}
Salvo disposi\c{c}\~{a}o em contr\'{a}rio, este trabalho foi licenciado com uma
Licen\c{c}a Creative Commons Atribui\c{c}\~{a}o - CompartilhaIgual 3.0 N\~{a}o
Adaptada. Para ver uma c\'{o}pia desta licen\c{c}a, visite
http://creativecommons.org/licenses/by-sa/3.0/.
\begin{center}
    \includegraphics{../figuras/cc-by-sa.png}
\end{center}
\newpage
\section{Fatora\c{c}\~{a}o de Cholesky}
\begin{defi}[Transposição]
    Dada um matriz $A \in \mathbb{R}^{n \times m} = [a_{ij}]$ a transposição
    dessa matriz corresponde a matriz $[a_{ji}]$ que é denotada por $A^t$.
\end{defi}
\begin{defi}[Matriz simétrica]
    Uma matriz é dita simétrica se, e somente se, $A = A^t$.
\end{defi}
\begin{defi}[Matriz definida positiva]
    Uma matriz $A \in \mathbb{R}^{n \times n}$ é definida positiva se, e somente
    se, $x^t A x > 0$ para todo $x \in \mathbb{R}^n$ não nulo.
\end{defi}

Matrizes simétrica definida positiva aparecem em vários problemas práticos e
para estas é possível provar os teoremas abaixo.
\begin{teo}
    Se $A$ é simétrica definida positiva, então $A$ é não singular (e portanto o
    sistema linear $A x = b$ possue solução única).
\end{teo}
\begin{teo}
    Seja $M \in \mathbb{R}^{m \times n}$ uma matriz qualquer e $A = M^t M$.
    Então $A$ é simétrica definida positiva.
\end{teo}

Motivado pela ocorrência das matrizes simétrica definida positiva e pelo fato
desta ser não singular, ``construi-se'' uma fatoração que vale apenas para essas
matrizes,a Fatoração de Cholesky, que é resumida no teorema a seguir.
\begin{teo}
    Seja $A$ definda positiva. Então $A$ pode ser decomposta de maneira única no
    produto $A = R R^t$ onde $R$ é uma matriz triangular inferior e todos os
    elementos da diagonal principal de $R$ são positivos.
\end{teo}

Saber, pela definição, se uma matriz $A \in \mathrm{R}^{n \times n}$ simétrica
qualquer é definida positiva costuma ser muito trabalhoso se não impossível. Por
esse motivo, costuma-se tentar calcular o fator de Cholesky para qualquer matriz
simétrica (e essa é uma das formas de descobrir se a matriz é definda positiva).

\section{Experimentos Computacionais}
Para investigar a Fatoração de Cholesky implementou-se as funções presentes no
Código~\ref{code:mt404_p05q01}.

Os resultados obtidos podem ser agrupados em três grupos representados pelas
Tabelas~\ref{tab:res_1},\ref{tab:res_2}~e~\ref{tab:res_3} que diferem-se pelos
casos em que não foi possível calcular o fator de Cholesky (quando o erro e o
resíduo são Inf).
\begin{table}[!htb]
    \centering
    \caption{Informações obtidas nos testes computacionais (1).}
    \label{tab:res_1}
    \csvautotabular{src/mt404_p05q01-1.csv}
\end{table}
\begin{table}[!htb]
    \centering
    \caption{Informações obtidas nos testes computacionais (2).}
    \label{tab:res_2}
    \csvautotabular{src/mt404_p05q01-2.csv}
\end{table}
\begin{table}[!htb]
    \centering
    \caption{Informações obtidas nos testes computacionais (3).}
    \label{tab:res_3}
    \csvautotabular{src/mt404_p05q01-3.csv}
\end{table}

As matrizes do tipo 4 são matrizes simétricas iguais a $G G^t$, sendo que
$G$ é uma matriz triangular inferior cujos elementos não nulos são valores
aleatórios entre 0 e 1. E as matrizes do tipo 5 diferem das matrizes do tipo 4
pelos elementos na diagonal da matriz $G$ serem aleatórios entre 1 e 2.

Como pode ser observado pelas tabelas, a matriz do tipo 3 falhou pois possui
pelo menos um elemento negativo na diagonal e algumas matrizes do tipo 4
falharam porque eram matrizes numericamente singulares (como os elementos da
diagonal de $G$ são maiores que zero e menores que um e $G$ é diagonal o
determinante de $G$ pode ser muito próximo de zero).

Pelas tabelas, observa-se que o erro relativo costuma ser próximo e
o resíduo aumenta com o aumento da dimensão da matriz.

\section{Matriz de Hilbert e a Fatoração de Cholesky}
A matriz de Hilbert de ordem $n$, $H^{(n)}$, é dada por
\begin{align*}
    H^{(n)}_{ij} = 1 / (i + j - 1).
\end{align*}

Pela expressão anterior, verifica-se que a matriz de Hilbert é simétrica e ao
analisar
\begin{align*}
    H^{(1)} &= \begin{bmatrix}
        1
    \end{bmatrix}, & H^{(2)} &= \begin{bmatrix}
        1 & 1/2 \\
        1/2 & 1/3
    \end{bmatrix}, & H^{(3)} &= \begin{bmatrix}
        1 & 1/2 & 1/3 \\
        1/2 & 1/3 & 1/4 \\
        1/3 & 1/4 & 1/5
    \end{bmatrix},
\end{align*}
verifica-se que $H^{(1)}$, $H^{(2)}$ e $H^{(3)}$ são simétrica definida
positiva. Como $H^{(1)}$, $H^{(2)}$ e $H^{(3)}$ são simétrica definida positiva
gostariamos de saber se as demais matrizes de Hilbert também são. Na literatura,
encontramos que as matrizes de Hilbert são simétricas definidas positivas e
também mal condicionadas. 

Como a matriz de Hilbert é simétrica definida positiva e mal condicionada,
gostariamos de saber a partir de quando (de que ordem) a mal condição da matriz
impossibilita encontrar, numericamente, a Fatoração de Cholesky da matriz de
Hilbert, i.e., a partir de que ordem a matriz de Hilbert não é mais definida
positiva.

Para encontrar a resposta a ``pergunta'' anterior, escreveu-se o
Código~\ref{code:mt404_p05q02} que executa um loop que a cada iteração aumenta a
ordem da matriz de Hilbert e em todas as iterações tenta calcular o fator de
Cholesky da matriz de Hilbert. Quando ocorrer do fator de Cholesky não ser
calculado, o loop é interrompido e a dimesão da matriz informada.

Ao executar o Código~\ref{code:mt404_p05q02} obtemos como resposta a
``pergunta'' que a maior dimensão para a qual é possível calcular o fator de
Cholesky é 12.

\section{C\'{o}digos}
A seguir encontra-se os c\'{o}digos desenvolvidos neste projeto. Todos os c\'{o}digos
foram testados utilizando o GNU Octave em sua vers\~{a}o
3.2.4\footnote{Acredita-se que os c\'{o}digos sejam compat\'{i}eis com o MATLAB
embora n\~{a}o tenham sido testados nesse ambiente.} e encontram-se
dispon\'{i}veis em \url{https://github.com/r-gaia-cs/mt404-2012s2}.
\lstinputlisting[style=codes, caption={Matrizes aleatórias},
label={code:mt404_p05q01}]{src/mt404_p05q01.m}
\lstinputlisting[style=codes, caption={Matriz de Hilbert},
label={code:mt404_p05q02}]{src/mt404_p05q02.m}
\end{document}
