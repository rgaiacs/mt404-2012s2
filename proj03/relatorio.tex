% Filename: relatorio.tex
%
% This code is part of 'MT404 2012s2 - Proj03'
% 
% Description: Relatorio do Projeto No.3 de MT404 2012s2.
% 
% Created: 22.08.12 07:25:12 AM
% Last Change: 22.08.12 07:25:12 AM
% 
% Author:
% - Raniere Silva, <r.gaia.cs@gmail.com>
% 
% Copyright (c) 2012, Raniere Silva. All rights reserved.
% 
% Except otherwise noted, this work is licensed under the Creative
% Commons Attribution-ShareAlike 3.0 Unported License. To view a copy of
% this license, visit http://creativecommons.org/licenses/by-sa/3.0/ or
% send a letter to Creative Commons, 444 Castro Street, Suite 900,
% Mountain View, California, 94041, USA.
%
% This work is distributed in the hope that it will be useful, but
% WITHOUT ANY WARRANTY; without even the implied warranty of
% MERCHANTABILITY or FITNESS FOR A PARTICULAR PURPOSE.
%
\documentclass[12pt,a4paper]{article}
% Filename: packages.tex
% 
% This code is part of 'MT404 2012s2'
% 
% Description: This file corresponds to the packages used.
% 
% Created: 22.08.12 07:25:12 AM
% Last Change: 22.08.12 08:23:08 PM
% 
% Authors:
% - Raniere Silva (2012): initial version
% 
% Copyright (c) 2012 Raniere Silva <r.gaia.cs@gmail.com>
% 
% This work is licensed under the Creative Commons Attribution-ShareAlike 3.0 Unported License. To view a copy of this license, visit http://creativecommons.org/licenses/by-sa/3.0/ or send a letter to Creative Commons, 444 Castro Street, Suite 900, Mountain View, California, 94041, USA.
%
% This work is distributed in the hope that it will be useful, but WITHOUT ANY WARRANTY; without even the implied warranty of MERCHANTABILITY or FITNESS FOR A PARTICULAR PURPOSE.
%
\usepackage[utf8]{inputenc}
\usepackage[T1]{fontenc}

% Configura\c{c}\~{o}es regionais
\usepackage[top=3cm,left=2cm,right=2cm,bottom=3cm]{geometry}
\usepackage[brazil]{babel}
\usepackage{indentfirst}

% Pacotes matem\'{a}ticos
\usepackage{amsmath}
\usepackage{amsthm}
\usepackage{amsfonts}
\usepackage{amssymb}
% Customiza\c{c}\~{a}o do pacote amsmath
\newtheorem{defi}{Definição}
\newtheorem{prop}{Proposição}
\allowdisplaybreaks[4]

% Pacotes gr\'{a}ficos
\usepackage{graphicx}
\usepackage{tikz}

% Links
\usepackage{url}
\usepackage{hyperref}

% Algoritmos
\usepackage{algorithmic} % Pacote para algoritmos
\usepackage{algorithm} % Pacote para algoritmos
% Customiza\c{c}\~{a}o do pacote algorithm
\floatname{algorithm}{Algoritmo}
\algsetup{linenosize=\small}
\renewcommand{\algorithmicrequire}{\textbf{Entrada:}}
\renewcommand{\algorithmicensure}{\textbf{Saída:}}
\renewcommand{\algorithmicend}{\textbf{fim}}
\renewcommand{\algorithmicif}{\textbf{se}}
\renewcommand{\algorithmicthen}{\textbf{ent\~{a}o}}
\renewcommand{\algorithmicelse}{\textbf{caso contr\'{a}rio}}
\renewcommand{\algorithmicendif}{\algorithmicend}
\renewcommand{\algorithmicfor}{\textbf{para}}
\renewcommand{\algorithmicforall}{\textbf{para todo}}
\renewcommand{\algorithmicdo}{\textbf{fa\c{c}a}}
\renewcommand{\algorithmicendfor}{\algorithmicend}
\renewcommand{\algorithmicwhile}{\textbf{enquanto}}
\renewcommand{\algorithmicendwhile}{\algorithmicend}
\renewcommand{\algorithmicrepeat}{\textbf{repita}}
\renewcommand{\algorithmicuntil}{\textbf{at\'{e}}}
\renewcommand{\algorithmicreturn}{\textbf{retorne}}
\renewcommand{\algorithmiccomment}[1]{\hspace{2em}/* #1 */}

% C\'{o}digos
\usepackage{listings} % Pacote para c\'{o}digos
\usepackage{textcomp} % Para aspas simples
\renewcommand{\lstlistingname}{C\'{o}digo}
\lstdefinestyle{codes}{ %
language=Octave,                % the language of the code
frame=single,                   % frame around the code
basicstyle=\ttfamily\small,     % the size of the fonts that are used for the code
columns=flexible,               % Correct unexpected spaces in strings when copy-and-past
numbers=left,                   % where to put the line-numbers
numberstyle=\footnotesize,      % the size of the fonts that are used for the line-numbers
stepnumber=5,                   % the step between two line-numbers.
numbersep=5pt,                  % how far the line-numbers are from the code
% backgroundcolor=\color{white},  % choose the background color. You must add \usepackage{color}
showspaces=false,               % show spaces adding particular underscores
showstringspaces=false,         % underline spaces within strings
% showtabs=false,                 % show tabs within strings adding particular underscores
% frame=single,                   % adds a frame around the code
tabsize=4,                      % sets default tabsize to 2 spaces
captionpos=t,                   % sets the caption-position to bottom
breaklines=true,                % sets automatic line breaking
breakatwhitespace=false,        % sets if automatic breaks should only happen at whitespace
% caption={\texttt{\lstname}},    % show the filename of files included with \lstinputlisting;
% escapeinside={\%*}{*)},         % if you want to add a comment within your code
% morekeywords={#},               % if you want to add more keywords to the set
upquote=true,                   % Correct single quote
}
\lstdefinestyle{outputs}{ %
language=Octave,                % the language of the code
% frame=single,                   % frame around the code
basicstyle=\ttfamily\small,     % the size of the fonts that are used for the code
columns=flexible,               % Correct unexpected spaces in strings when copy-and-past
% numbers=left,                   % where to put the line-numbers
% numberstyle=\footnotesize,      % the size of the fonts that are used for the line-numbers
% stepnumber=5,                   % the step between two line-numbers.
% numbersep=5pt,                  % how far the line-numbers are from the code
% backgroundcolor=\color{white},  % choose the background color. You must add \usepackage{color}
showspaces=false,               % show spaces adding particular underscores
showstringspaces=false,         % underline spaces within strings
% showtabs=false,                 % show tabs within strings adding particular underscores
% frame=single,                   % adds a frame around the code
tabsize=4,                      % sets default tabsize to 2 spaces
% captionpos=t,                   % sets the caption-position to bottom
breaklines=true,                % sets automatic line breaking
breakatwhitespace=false,        % sets if automatic breaks should only happen at whitespace
% caption={\texttt{\lstname}},    % show the filename of files included with \lstinputlisting;
% escapeinside={\%*}{*)},         % if you want to add a comment within your code
% morekeywords={#}                % if you want to add more keywords to the set
upquote=true,                   % Correct single quote
}

% Novos comandos

\begin{document}
\title{Projeto No.3 de MT404}
\author{Raniere Silva \\ ra092767}
\maketitle
\begin{abstract}
    Este \'{e} o projeto no.3 da disciplina MT404 - M\'{e}todos Computacionais
    de \'{A}lgebra Linear. Neste projeto 
\end{abstract}
\tableofcontents
\lstlistoflistings
\section*{Licen\c{c}a}
Salvo disposi\c{c}\~{a}o em contr\'{a}rio, este trabalho foi licenciado com uma
Licen\c{c}a Creative Commons Atribui\c{c}\~{a}o - CompartilhaIgual 3.0 N\~{a}o
Adaptada. Para ver uma c\'{o}pia desta licen\c{c}a, visite
http://creativecommons.org/licenses/by-sa/3.0/.
\begin{center}
    \includegraphics{../figuras/cc-by-sa.png}
\end{center}
\newpage
\section{An\'{a}lise de sensibilidade}
Seja $A \in \mathbb{R}^{n \times n}$ e $x, b \in \mathbb{R}^n$. Considere o
sistema linear $A x = b$ e a solu\c{c}\~{a}o $\hat{x}$ obtida atrav\'{e}s de
m\'{e}todos num\'{e}ricos.

\begin{defi}
    O res\'{i}duo de $\hat{x}$ em rela\c{c}\~{a}o a $A x = b$ \'{e} o vetor $A
    \hat{x} - b$.
\end{defi}

\section{Experimentos Computacionais}
O experimento computacional desta atividade consistiu em utilizar a
fun\c{c}\~{a}o nativa do GNU Octave para resolver sistemas lineares e calcular o
res\'{i}duo, o n\'{u}mero de condi\c{c}\~{a}o da matriz e um limitante para o
erro relativo da solu\c{c}\~{a}o calculada.

Para a matrix $A \in \mathbb{R}^{n \times n}$ dada por
\begin{align*}
    a_{ii} &= 0.5 + (0.1 / n) i & i &= 1, \ldots, n - 1, \\
    a_{nn} &= 0.6, \\
    a_{i,i+1} &= -1 & i &= 1, \ldots, n - 1
\end{align*}
obtivemos a sa\'{i}da que corresponde a Tabela~
\lstinputlisting[style=outputs, nolol=true]{src/mt404_p03q01.out}
\begin{table}[!htb]
    \centering
    \caption{Resultados referentes a matriz de banda.}
    \label{tab:res_matriz_banda}
    \begin{tabular}{|c|c|c|c|c|}
        \hline
        $n$ & Lado Direito & $\kappa(A)$ & Res\'{i}duo & Limitante do erro \\ \hline
        10 & $b$ & 1.402391e+03 & 5.684342e-14 & 7.971673e-11 \\ \hline
        10 & $\tilde{b}$ & 1.402391e+03 & 0.000000e+00 & 0.000000e+00 \\
        \hline
        100 & $b$ & 3.839762e+26 & 2.684355e+08 & 1.030728e+35 \\ \hline
        100 & $\tilde{b}$ & 3.839762e+26 & 1.639128e-10 & 1.048977e+17
        \\ \hline
    \end{tabular}
\end{table}


Para a matriz de Hilbert
\lstinputlisting[style=outputs, nolol=true]{src/mt404_p03q02.out}
\begin{table}[!htb]
    \centering
    \caption{Resultados referentes a matriz de Hilbert.}
    \label{tab:res_matriz_banda}
    \begin{tabular}{|c|c|c|c|}
        \hline
        $n$ & $\kappa(A)$ & Res\'{i}duo & Limitante do erro \\ \hline
        5 & 4.766073e+05 & 2.220446e-16 & 4.634806e-11 \\ \hline
        10 & 1.602500e+13 & 2.220446e-16 & 1.214853e-03 \\ \hline
        50 & 1.017747e+19 & 3.552714e-15 & 8.036452e+03 \\ \hline
        100 & 8.156265e+19 & 6.217249e-15 & 9.775562e+04 \\ \hline
        1000 & 1.926857e+21 & 4.085621e-14 & 1.051692e+07 \\ \hline
    \end{tabular}
\end{table}

\section{C\'{o}digos}
A seguir encontra-se os códigos desenvolvidos neste projeto. Todos os códigos
foram testados utilizando o GNU Octave em sua versão 3.2.4\footnote{Acredita-se
que os códigos sejam compatíveis com o MATLAB embora não tenham sido testados
nesse ambiente.} e encontram-se disponíveis em
\url{https://github.com/r-gaia-cs/mt404-2012s2}.
\lstinputlisting[style=codes, caption={An\'{a}lise de sensibilidade},
label={code:mt404_p03q01}]{src/mt404_p03q01.m}
\lstinputlisting[style=codes, caption={An\'{a}lise de sensibilidade para matriz
de Hilbert}, label={code:mt404_p03q01}]{src/mt404_p03q01.m}
\end{document}
