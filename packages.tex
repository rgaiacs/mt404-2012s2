% Filename: packages.tex
% 
% This code is part of 'MT404 2012s2'
% 
% Description: This file corresponds to the packages used.
% 
% Created: 22.08.12 07:25:12 AM
% Last Change: 22.08.12 08:23:08 PM
% 
% Authors:
% - Raniere Silva (2012): initial version
% 
% Copyright (c) 2012 Raniere Silva <r.gaia.cs@gmail.com>
% 
% This work is licensed under the Creative Commons Attribution-ShareAlike 3.0 Unported License. To view a copy of this license, visit http://creativecommons.org/licenses/by-sa/3.0/ or send a letter to Creative Commons, 444 Castro Street, Suite 900, Mountain View, California, 94041, USA.
%
% This work is distributed in the hope that it will be useful, but WITHOUT ANY WARRANTY; without even the implied warranty of MERCHANTABILITY or FITNESS FOR A PARTICULAR PURPOSE.
%
\usepackage[utf8]{inputenc}
\usepackage[T1]{fontenc}

% Configura\c{c}\~{o}es regionais
\usepackage[top=3cm,left=2cm,right=2cm,bottom=3cm]{geometry}
\usepackage[brazil]{babel}
\usepackage{indentfirst}

% Pacotes matem\'{a}ticos
\usepackage{amsmath}
\usepackage{amsthm}
\usepackage{amsfonts}
\usepackage{amssymb}
% Customiza\c{c}\~{a}o do pacote amsmath
\newtheorem{defi}{Definição}
\newtheorem{prop}{Proposição}
\allowdisplaybreaks[4]

% Pacotes gr\'{a}ficos
\usepackage{graphicx}
\usepackage{tikz}

% Links
\usepackage{url}
\usepackage{hyperref}

% Algoritmos
\usepackage{algorithmic} % Pacote para algoritmos
\usepackage{algorithm} % Pacote para algoritmos
% Customiza\c{c}\~{a}o do pacote algorithm
\floatname{algorithm}{Algoritmo}
\algsetup{linenosize=\small}
\renewcommand{\algorithmicrequire}{\textbf{Entrada:}}
\renewcommand{\algorithmicensure}{\textbf{Saída:}}
\renewcommand{\algorithmicend}{\textbf{fim}}
\renewcommand{\algorithmicif}{\textbf{se}}
\renewcommand{\algorithmicthen}{\textbf{ent\~{a}o}}
\renewcommand{\algorithmicelse}{\textbf{caso contr\'{a}rio}}
\renewcommand{\algorithmicendif}{\algorithmicend}
\renewcommand{\algorithmicfor}{\textbf{para}}
\renewcommand{\algorithmicforall}{\textbf{para todo}}
\renewcommand{\algorithmicdo}{\textbf{fa\c{c}a}}
\renewcommand{\algorithmicendfor}{\algorithmicend}
\renewcommand{\algorithmicwhile}{\textbf{enquanto}}
\renewcommand{\algorithmicendwhile}{\algorithmicend}
\renewcommand{\algorithmicrepeat}{\textbf{repita}}
\renewcommand{\algorithmicuntil}{\textbf{at\'{e}}}
\renewcommand{\algorithmicreturn}{\textbf{retorne}}
\renewcommand{\algorithmiccomment}[1]{\hspace{2em}/* #1 */}

% C\'{o}digos
\usepackage{listings} % Pacote para c\'{o}digos
\usepackage{textcomp} % Para aspas simples
\renewcommand{\lstlistingname}{C\'{o}digo}
\lstdefinestyle{codes}{ %
language=Octave,                % the language of the code
frame=single,                   % frame around the code
basicstyle=\ttfamily\small,     % the size of the fonts that are used for the code
columns=flexible,               % Correct unexpected spaces in strings when copy-and-past
numbers=left,                   % where to put the line-numbers
numberstyle=\footnotesize,      % the size of the fonts that are used for the line-numbers
stepnumber=5,                   % the step between two line-numbers.
numbersep=5pt,                  % how far the line-numbers are from the code
% backgroundcolor=\color{white},  % choose the background color. You must add \usepackage{color}
% showspaces=false,               % show spaces adding particular underscores
% showstringspaces=false,         % underline spaces within strings
% showtabs=false,                 % show tabs within strings adding particular underscores
% frame=single,                   % adds a frame around the code
tabsize=4,                      % sets default tabsize to 2 spaces
captionpos=t,                   % sets the caption-position to bottom
breaklines=true,                % sets automatic line breaking
breakatwhitespace=false,        % sets if automatic breaks should only happen at whitespace
% caption={\texttt{\lstname}},    % show the filename of files included with \lstinputlisting;
% escapeinside={\%*}{*)},         % if you want to add a comment within your code
% morekeywords={#},               % if you want to add more keywords to the set
upquote=true,                   % Correct single quote
}
\lstdefinestyle{outputs}{ %
language=Octave,                % the language of the code
% frame=single,                   % frame around the code
basicstyle=\ttfamily\small,     % the size of the fonts that are used for the code
columns=flexible,               % Correct unexpected spaces in strings when copy-and-past
% numbers=left,                   % where to put the line-numbers
% numberstyle=\footnotesize,      % the size of the fonts that are used for the line-numbers
% stepnumber=5,                   % the step between two line-numbers.
% numbersep=5pt,                  % how far the line-numbers are from the code
% backgroundcolor=\color{white},  % choose the background color. You must add \usepackage{color}
% showspaces=false,               % show spaces adding particular underscores
% showstringspaces=false,         % underline spaces within strings
% showtabs=false,                 % show tabs within strings adding particular underscores
% frame=single,                   % adds a frame around the code
tabsize=4,                      % sets default tabsize to 2 spaces
% captionpos=t,                   % sets the caption-position to bottom
breaklines=true,                % sets automatic line breaking
breakatwhitespace=false,        % sets if automatic breaks should only happen at whitespace
% caption={\texttt{\lstname}},    % show the filename of files included with \lstinputlisting;
% escapeinside={\%*}{*)},         % if you want to add a comment within your code
% morekeywords={#}                % if you want to add more keywords to the set
upquote=true,                   % Correct single quote
}

% Novos comandos
