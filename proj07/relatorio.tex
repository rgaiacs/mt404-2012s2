% Filename: relatorio.tex
%
% This code is part of 'MT404 2012s2 - Proj07'
% 
% Description: Relatorio do Projeto No.7 de MT404 2012s2.
% 
% Created: 22.08.12 07:25:12 AM
% Last Change: 22.08.12 07:25:12 AM
% 
% Author:
% - Raniere Silva, <r.gaia.cs@gmail.com>
% 
% Copyright (c) 2012, Raniere Silva. All rights reserved.
% 
% Except otherwise noted, this work is licensed under the Creative
% Commons Attribution-ShareAlike 3.0 Unported License. To view a copy of
% this license, visit http://creativecommons.org/licenses/by-sa/3.0/ or
% send a letter to Creative Commons, 444 Castro Street, Suite 900,
% Mountain View, California, 94041, USA.
%
% This work is distributed in the hope that it will be useful, but
% WITHOUT ANY WARRANTY; without even the implied warranty of
% MERCHANTABILITY or FITNESS FOR A PARTICULAR PURPOSE.
%
\documentclass[12pt,a4paper]{article}
% Filename: packages.tex
% 
% This code is part of 'MT404 2012s2'
% 
% Description: This file corresponds to the packages used.
% 
% Created: 22.08.12 07:25:12 AM
% Last Change: 22.08.12 08:23:08 PM
% 
% Authors:
% - Raniere Silva (2012): initial version
% 
% Copyright (c) 2012 Raniere Silva <r.gaia.cs@gmail.com>
% 
% This work is licensed under the Creative Commons Attribution-ShareAlike 3.0 Unported License. To view a copy of this license, visit http://creativecommons.org/licenses/by-sa/3.0/ or send a letter to Creative Commons, 444 Castro Street, Suite 900, Mountain View, California, 94041, USA.
%
% This work is distributed in the hope that it will be useful, but WITHOUT ANY WARRANTY; without even the implied warranty of MERCHANTABILITY or FITNESS FOR A PARTICULAR PURPOSE.
%
\usepackage[utf8]{inputenc}
\usepackage[T1]{fontenc}

% Configura\c{c}\~{o}es regionais
\usepackage[top=3cm,left=2cm,right=2cm,bottom=3cm]{geometry}
\usepackage[brazil]{babel}
\usepackage{indentfirst}

% Pacotes matem\'{a}ticos
\usepackage{amsmath}
\usepackage{amsthm}
\usepackage{amsfonts}
\usepackage{amssymb}
% Customiza\c{c}\~{a}o do pacote amsmath
\newtheorem{defi}{Definição}
\newtheorem{prop}{Proposição}
\allowdisplaybreaks[4]

% Pacotes gr\'{a}ficos
\usepackage{graphicx}
\usepackage{tikz}

% Links
\usepackage{url}
\usepackage{hyperref}

% Algoritmos
\usepackage{algorithmic} % Pacote para algoritmos
\usepackage{algorithm} % Pacote para algoritmos
% Customiza\c{c}\~{a}o do pacote algorithm
\floatname{algorithm}{Algoritmo}
\algsetup{linenosize=\small}
\renewcommand{\algorithmicrequire}{\textbf{Entrada:}}
\renewcommand{\algorithmicensure}{\textbf{Saída:}}
\renewcommand{\algorithmicend}{\textbf{fim}}
\renewcommand{\algorithmicif}{\textbf{se}}
\renewcommand{\algorithmicthen}{\textbf{ent\~{a}o}}
\renewcommand{\algorithmicelse}{\textbf{caso contr\'{a}rio}}
\renewcommand{\algorithmicendif}{\algorithmicend}
\renewcommand{\algorithmicfor}{\textbf{para}}
\renewcommand{\algorithmicforall}{\textbf{para todo}}
\renewcommand{\algorithmicdo}{\textbf{fa\c{c}a}}
\renewcommand{\algorithmicendfor}{\algorithmicend}
\renewcommand{\algorithmicwhile}{\textbf{enquanto}}
\renewcommand{\algorithmicendwhile}{\algorithmicend}
\renewcommand{\algorithmicrepeat}{\textbf{repita}}
\renewcommand{\algorithmicuntil}{\textbf{at\'{e}}}
\renewcommand{\algorithmicreturn}{\textbf{retorne}}
\renewcommand{\algorithmiccomment}[1]{\hspace{2em}/* #1 */}

% C\'{o}digos
\usepackage{listings} % Pacote para c\'{o}digos
\usepackage{textcomp} % Para aspas simples
\renewcommand{\lstlistingname}{C\'{o}digo}
\lstdefinestyle{codes}{ %
language=Octave,                % the language of the code
frame=single,                   % frame around the code
basicstyle=\ttfamily\small,     % the size of the fonts that are used for the code
columns=flexible,               % Correct unexpected spaces in strings when copy-and-past
numbers=left,                   % where to put the line-numbers
numberstyle=\footnotesize,      % the size of the fonts that are used for the line-numbers
stepnumber=5,                   % the step between two line-numbers.
numbersep=5pt,                  % how far the line-numbers are from the code
% backgroundcolor=\color{white},  % choose the background color. You must add \usepackage{color}
showspaces=false,               % show spaces adding particular underscores
showstringspaces=false,         % underline spaces within strings
% showtabs=false,                 % show tabs within strings adding particular underscores
% frame=single,                   % adds a frame around the code
tabsize=4,                      % sets default tabsize to 2 spaces
captionpos=t,                   % sets the caption-position to bottom
breaklines=true,                % sets automatic line breaking
breakatwhitespace=false,        % sets if automatic breaks should only happen at whitespace
% caption={\texttt{\lstname}},    % show the filename of files included with \lstinputlisting;
% escapeinside={\%*}{*)},         % if you want to add a comment within your code
% morekeywords={#},               % if you want to add more keywords to the set
upquote=true,                   % Correct single quote
}
\lstdefinestyle{outputs}{ %
language=Octave,                % the language of the code
% frame=single,                   % frame around the code
basicstyle=\ttfamily\small,     % the size of the fonts that are used for the code
columns=flexible,               % Correct unexpected spaces in strings when copy-and-past
% numbers=left,                   % where to put the line-numbers
% numberstyle=\footnotesize,      % the size of the fonts that are used for the line-numbers
% stepnumber=5,                   % the step between two line-numbers.
% numbersep=5pt,                  % how far the line-numbers are from the code
% backgroundcolor=\color{white},  % choose the background color. You must add \usepackage{color}
showspaces=false,               % show spaces adding particular underscores
showstringspaces=false,         % underline spaces within strings
% showtabs=false,                 % show tabs within strings adding particular underscores
% frame=single,                   % adds a frame around the code
tabsize=4,                      % sets default tabsize to 2 spaces
% captionpos=t,                   % sets the caption-position to bottom
breaklines=true,                % sets automatic line breaking
breakatwhitespace=false,        % sets if automatic breaks should only happen at whitespace
% caption={\texttt{\lstname}},    % show the filename of files included with \lstinputlisting;
% escapeinside={\%*}{*)},         % if you want to add a comment within your code
% morekeywords={#}                % if you want to add more keywords to the set
upquote=true,                   % Correct single quote
}

% Novos comandos

\usepackage{csvsimple}
\begin{document}
\title{Projeto No.7 de MT404}
\author{Raniere Silva \\ ra092767}
\maketitle
\begin{abstract}
    Este \'{e} o projeto no.7 da disciplina MT404 - M\'{e}todos Computacionais
    de \'{A}lgebra Linear. Neste projeto utilizou-se a subrotina dgels da
    biblioteca LAPACK para resolver alguns problemas de quadrados mínimos.
\end{abstract}
\tableofcontents
\lstlistoflistings
\section*{Licen\c{c}a}
Salvo disposi\c{c}\~{a}o em contr\'{a}rio, este trabalho foi licenciado com uma
Licen\c{c}a Creative Commons Atribui\c{c}\~{a}o - CompartilhaIgual 3.0 N\~{a}o
Adaptada. Para ver uma c\'{o}pia desta licen\c{c}a, visite
http://creativecommons.org/licenses/by-sa/3.0/.
\begin{center}
    \includegraphics{../figuras/cc-by-sa.png}
\end{center}
\newpage
\section{Quadrados Mínimos e Fatoração QR}
Considere o sistema linear sobredeterminado
\begin{align*}
    A x & = b,
\end{align*}
onde $A \in \mathbb{R}^{m \times n}$ e $b \in \mathbb{R}^m$. O problema de
quadrados mínimos para o sistama anterior é definido como encontrar $x \in
\mathbb{R}^n$ para o qual $\| A x - b \|_2$ é mínimo.

A Fatoração QR consiste em encontrar matrizes $Q \in \mathbb{R}^{m \times n}$ e
$R \in \mathbb{R}^{n \times n}$ tal que $A = QR$, $Q$ é uma matriz
ortogonal e $R$ é uma matriz triangular superior.

O sistema linear $A x = b$ pode ser transformado em
\begin{align*}
    Q^T A x &= Q^T b.
\end{align*}
Então
\begin{align*}
    Q^T A &= \begin{bmatrix}
        \hat{R} \\
        0
    \end{bmatrix}.
\end{align*}
Considerando que $A$ possue posto completo e que
\begin{align*}
    Q^T b &= \begin{bmatrix}
        \hat{c} \\
        \hat{d}
    \end{bmatrix},
\end{align*}
temos que
\begin{align*}
    A x - b &= \begin{bmatrix}
        \hat{R} \\
        0
    \end{bmatrix} x - \begin{bmatrix}
        \hat{c} \\
        \hat{d}
    \end{bmatrix} \\
    &= \begin{bmatrix}
        \hat{R} x - \hat{c} \\
        \hat{d}
    \end{bmatrix}.
\end{align*}
Como $A$ possue posto completo, então $\hat{R}$ possue posto compelto e
portanto existe $x^*$ tal que $\hat{R} x^* - \hat{c} = 0$. Logo, a
solução para o problema de quadrados mínimos é $\| \hat{d} \|$.

\section{Resultados Computacionais}
Para os testes computacionais, utilizou-se a subrotina dgels, da
biblioteca LAPACK, que dentre outras funcionalidades resolve sistemas
lineares sobredeterminados.

Para os testes computacionais, utilizou-se os sistemas lineares
correspondentes a:
\begin{itemize}
    \item para $n$ variável e $m \geq n$:
        \begin{align}
            a_i^T x - b_i &= x_i - 2/m \left( \sum_{j = 1}^n x_j
            \right) - 1, & 1 \leq i \leq n, \notag \\
            a_i^T x - b_i &= -2/m \left( \sum_{j = 1}^n x_j \right) - 1, & 1
            < i \leq m,
            \label{eq:test01}
        \end{align}
    \item para $n$ variável e $m \geq n$:
        \begin{align}
            a_i^T x - b_i &= i \left( \sum_{j = 1}^n j x_j \right) - 1, & 1 \leq
            i \leq m,
            \label{eq:test02}
        \end{align}
    \item para $n$ variável e $m \geq n$:
        \begin{align}
            a_1^T x - b_1 &= -1, \notag \\
            a_m^T x - b_m &= -1
            \label{eq:test03} \\
            a_i^T x - b_i &= (i - 1) \left( \sum_{j = 2}^{n - 1} j x_j
            \right) - 1, & 2 \leq i < m. \notag
        \end{align}
\end{itemize}

Os resultados obtidos nos testes computacionais encontram-se nas
Tabelas~\ref{tab:res01},~\ref{tab:res02}~e~\ref{tab:res03}, onde --- indica
que aquele teste não foi realizado e *** que a subrotina degels falhou.
\begin{table}[!htb]
    \centering
    \caption{Valores do resíduo para \eqref{eq:test01} ao variar $m$ e $n$.}
    \label{tab:res01}
    \csvautotabular{src/test01.csv}
\end{table}
\begin{table}[!htb]
    \centering
    \caption{Valores do resíduo para \eqref{eq:test02} ao variar $m$ e $n$.}
    \label{tab:res02}
    \csvautotabular{src/test02.csv}
\end{table}
\begin{table}[!htb]
    \centering
    \caption{Valores do resíduo para \eqref{eq:test03} ao variar $m$ e $n$.}
    \label{tab:res03}
    \csvautotabular{src/test03.csv}
\end{table}

Pelos dados presentes nas
Tabelas~\ref{tab:res01},~\ref{tab:res02}~e~\ref{tab:res03} observamos que ao
aumentar o número de linhas da matriz $A$ ($m$) o resíduo aumenta e ao
aumentar o número de colunas da matriz $A$ ($n$) o resíduo diminui. Ambos os
comportamentos descritos anteriormente são explicados pelo caso
particular do problema de quadrados mínimos de ajustar uma reta a um
conjunto de pontos.

Nas Tabelas~\ref{tab:res01}~e~\ref{tab:res02} verifica-se que para $m = n$ o
resíduo é zero pois a matriz $A$ possue posto completo. Já na
Tabela~\ref{tab:res03} para $m = n$ a subrotina dgels falou devido as
equações
\begin{align*}
    a_1^T - b_1 &= -1, \\
    a_m^T - b_m &= -1
\end{align*}
que deixam o problema sem solução.

\section{C\'{o}digos}
A seguir encontra-se os c\'{o}digos desenvolvidos neste projeto e que
encontram-se disponíveis em \url{https://github.com/r-gaia-cs/mt404-2012s2}.

Todos os c\'{o}digos foram implementados em Fortran (utilizando a
biblioteca LAPACK\nocite{LAPACK}) e as instruções para
compilação encontram-se no Código~\ref{code:Makefile}. Para os testes
computacionais utilizou-se a versão do gfortran baseada no gcc 4.6.3.
\lstinputlisting[style=codes, caption={Rotina de teste},
label={code:mt404_p06}]{src/mt404_p07.f}
\lstinputlisting[style=codes, caption={Makefile},
label={code:Makefile}]{src/Makefile}

\addcontentsline{toc}{section}{Refer\^{e}ncia Bibliogr\'{a}fica}
\bibliographystyle{alpha}
\bibliography{../referencias}
\end{document}
