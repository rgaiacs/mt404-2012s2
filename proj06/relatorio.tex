% Filename: relatorio.tex
%
% This code is part of 'MT404 2012s2 - Proj06'
% 
% Description: Relatorio do Projeto No.6 de MT404 2012s2.
% 
% Created: 22.08.12 07:25:12 AM
% Last Change: 22.08.12 07:25:12 AM
% 
% Author:
% - Raniere Silva, <r.gaia.cs@gmail.com>
% 
% Copyright (c) 2012, Raniere Silva. All rights reserved.
% 
% Except otherwise noted, this work is licensed under the Creative
% Commons Attribution-ShareAlike 3.0 Unported License. To view a copy of
% this license, visit http://creativecommons.org/licenses/by-sa/3.0/ or
% send a letter to Creative Commons, 444 Castro Street, Suite 900,
% Mountain View, California, 94041, USA.
%
% This work is distributed in the hope that it will be useful, but
% WITHOUT ANY WARRANTY; without even the implied warranty of
% MERCHANTABILITY or FITNESS FOR A PARTICULAR PURPOSE.
%
\documentclass[12pt,a4paper]{article}
% Filename: packages.tex
% 
% This code is part of 'MT404 2012s2'
% 
% Description: This file corresponds to the packages used.
% 
% Created: 22.08.12 07:25:12 AM
% Last Change: 22.08.12 08:23:08 PM
% 
% Authors:
% - Raniere Silva (2012): initial version
% 
% Copyright (c) 2012 Raniere Silva <r.gaia.cs@gmail.com>
% 
% This work is licensed under the Creative Commons Attribution-ShareAlike 3.0 Unported License. To view a copy of this license, visit http://creativecommons.org/licenses/by-sa/3.0/ or send a letter to Creative Commons, 444 Castro Street, Suite 900, Mountain View, California, 94041, USA.
%
% This work is distributed in the hope that it will be useful, but WITHOUT ANY WARRANTY; without even the implied warranty of MERCHANTABILITY or FITNESS FOR A PARTICULAR PURPOSE.
%
\usepackage[utf8]{inputenc}
\usepackage[T1]{fontenc}

% Configura\c{c}\~{o}es regionais
\usepackage[top=3cm,left=2cm,right=2cm,bottom=3cm]{geometry}
\usepackage[brazil]{babel}
\usepackage{indentfirst}

% Pacotes matem\'{a}ticos
\usepackage{amsmath}
\usepackage{amsthm}
\usepackage{amsfonts}
\usepackage{amssymb}
% Customiza\c{c}\~{a}o do pacote amsmath
\newtheorem{defi}{Definição}
\newtheorem{prop}{Proposição}
\allowdisplaybreaks[4]

% Pacotes gr\'{a}ficos
\usepackage{graphicx}
\usepackage{tikz}

% Links
\usepackage{url}
\usepackage{hyperref}

% Algoritmos
\usepackage{algorithmic} % Pacote para algoritmos
\usepackage{algorithm} % Pacote para algoritmos
% Customiza\c{c}\~{a}o do pacote algorithm
\floatname{algorithm}{Algoritmo}
\algsetup{linenosize=\small}
\renewcommand{\algorithmicrequire}{\textbf{Entrada:}}
\renewcommand{\algorithmicensure}{\textbf{Saída:}}
\renewcommand{\algorithmicend}{\textbf{fim}}
\renewcommand{\algorithmicif}{\textbf{se}}
\renewcommand{\algorithmicthen}{\textbf{ent\~{a}o}}
\renewcommand{\algorithmicelse}{\textbf{caso contr\'{a}rio}}
\renewcommand{\algorithmicendif}{\algorithmicend}
\renewcommand{\algorithmicfor}{\textbf{para}}
\renewcommand{\algorithmicforall}{\textbf{para todo}}
\renewcommand{\algorithmicdo}{\textbf{fa\c{c}a}}
\renewcommand{\algorithmicendfor}{\algorithmicend}
\renewcommand{\algorithmicwhile}{\textbf{enquanto}}
\renewcommand{\algorithmicendwhile}{\algorithmicend}
\renewcommand{\algorithmicrepeat}{\textbf{repita}}
\renewcommand{\algorithmicuntil}{\textbf{at\'{e}}}
\renewcommand{\algorithmicreturn}{\textbf{retorne}}
\renewcommand{\algorithmiccomment}[1]{\hspace{2em}/* #1 */}

% C\'{o}digos
\usepackage{listings} % Pacote para c\'{o}digos
\usepackage{textcomp} % Para aspas simples
\renewcommand{\lstlistingname}{C\'{o}digo}
\lstdefinestyle{codes}{ %
language=Octave,                % the language of the code
frame=single,                   % frame around the code
basicstyle=\ttfamily\small,     % the size of the fonts that are used for the code
columns=flexible,               % Correct unexpected spaces in strings when copy-and-past
numbers=left,                   % where to put the line-numbers
numberstyle=\footnotesize,      % the size of the fonts that are used for the line-numbers
stepnumber=5,                   % the step between two line-numbers.
numbersep=5pt,                  % how far the line-numbers are from the code
% backgroundcolor=\color{white},  % choose the background color. You must add \usepackage{color}
showspaces=false,               % show spaces adding particular underscores
showstringspaces=false,         % underline spaces within strings
% showtabs=false,                 % show tabs within strings adding particular underscores
% frame=single,                   % adds a frame around the code
tabsize=4,                      % sets default tabsize to 2 spaces
captionpos=t,                   % sets the caption-position to bottom
breaklines=true,                % sets automatic line breaking
breakatwhitespace=false,        % sets if automatic breaks should only happen at whitespace
% caption={\texttt{\lstname}},    % show the filename of files included with \lstinputlisting;
% escapeinside={\%*}{*)},         % if you want to add a comment within your code
% morekeywords={#},               % if you want to add more keywords to the set
upquote=true,                   % Correct single quote
}
\lstdefinestyle{outputs}{ %
language=Octave,                % the language of the code
% frame=single,                   % frame around the code
basicstyle=\ttfamily\small,     % the size of the fonts that are used for the code
columns=flexible,               % Correct unexpected spaces in strings when copy-and-past
% numbers=left,                   % where to put the line-numbers
% numberstyle=\footnotesize,      % the size of the fonts that are used for the line-numbers
% stepnumber=5,                   % the step between two line-numbers.
% numbersep=5pt,                  % how far the line-numbers are from the code
% backgroundcolor=\color{white},  % choose the background color. You must add \usepackage{color}
showspaces=false,               % show spaces adding particular underscores
showstringspaces=false,         % underline spaces within strings
% showtabs=false,                 % show tabs within strings adding particular underscores
% frame=single,                   % adds a frame around the code
tabsize=4,                      % sets default tabsize to 2 spaces
% captionpos=t,                   % sets the caption-position to bottom
breaklines=true,                % sets automatic line breaking
breakatwhitespace=false,        % sets if automatic breaks should only happen at whitespace
% caption={\texttt{\lstname}},    % show the filename of files included with \lstinputlisting;
% escapeinside={\%*}{*)},         % if you want to add a comment within your code
% morekeywords={#}                % if you want to add more keywords to the set
upquote=true,                   % Correct single quote
}

% Novos comandos

\usepackage{csvsimple}
\begin{document}
\title{Projeto No.6 de MT404}
\author{Raniere Silva \\ ra092767  \and Julio Cesar \\ ra984581}
\maketitle
\begin{abstract}
    Este \'{e} o projeto no.6 da disciplina MT404 - M\'{e}todos Computacionais
    de \'{A}lgebra Linear. Neste projeto implementou-se em Fortran a Fatoração
    de Cholesky para uma matriz cheia e realizou-se alguns testes
    computacionais.
\end{abstract}
\tableofcontents
\lstlistoflistings
\section*{Licen\c{c}a}
Salvo disposi\c{c}\~{a}o em contr\'{a}rio, este trabalho foi licenciado com uma
Licen\c{c}a Creative Commons Atribui\c{c}\~{a}o - CompartilhaIgual 3.0 N\~{a}o
Adaptada. Para ver uma c\'{o}pia desta licen\c{c}a, visite
http://creativecommons.org/licenses/by-sa/3.0/.
\begin{center}
    \includegraphics{../figuras/cc-by-sa.png}
\end{center}
\newpage
\section{Fatora\c{c}\~{a}o de Cholesky}
A Fatoração de Cholesky é definida no teorema abaixo.
\begin{teo}
    Seja $A \in \mathbb{R}^{n \times n}$ uma matriz simétrica definda positiva.
    Então $A$ pode ser decomposta de maneira única no produto $A = R R^t$ onde
    $R$ é uma matriz triangular inferior e todos os elementos da diagonal
    principal de $R$ são positivos.
\end{teo}
Uma matriz é simétrica se, e somente se, $A = A^t$. E uma matriz $A \in
\mathbb{R}^{n \times n}$ é definida positiva se, e somente se, $x^t A x > 0$
para todo $x \in \mathbb{R}^n$ não nulo.

Saber, pela definição, se uma matriz $A \in \mathrm{R}^{n \times n}$ simétrica
qualquer é definida positiva costuma ser muito trabalhoso se não impossível. Por
esse motivo, costuma-se tentar calcular o fator de Cholesky para qualquer matriz
simétrica (e essa é uma das formas de descobrir se a matriz é definda
positiva).\footnote{A justificativa para essa afirmação encontram-se na maioria
dos livros de Álgebra Linear Numérica.}

\section{Experimentos Computacionais}
Para os experimentos computacionais implementou-se a Fatoração de Cholesky em
Fortran, ver Código~\ref{code:cholesky}, e várias outras rotinas auxiliares para
manipular matrizes e vetores, ver
Códigos~\ref{code:vector}~e~\ref{code:square_matrix}, respectivamente.

Os experimentos computacionais foram implementados no
Código~\ref{code:mt404_p06} e inspirados no Projeto No.5 de MT404
\cite{Raniere-2012-MT404Proj05}.

Como apresentado no Projeto No.5 de MT404, ao gerar uma matriz $G \in
\mathbb{R}^{n \times n}$ triangular inferior ``aleatória'' com elementos entre 0
e 1 e calcular $A = G G^t$ observa-se que a chance da matriz $A$ ser
numericamente indefinida é elevada. Uma solução para esse problema é adicionar
uma unidade a todos os elementos da diagonal principal de $G$ e depois calcular
$A = G G^t$. Por esse motivo, nos experimentos computacionais realizados a
matriz $G$ possui os elementos da diagonal principal entre 1 e 2.

Os dados obtidos nos experimentos computacionais são apresentados na
Tabela~\ref{tab:res}, os valores inf para o erro e resíduo indicam que a matriz
testada não é simétrica definida positiva. Comparando essa tabela com as tabelas
presentes no Projeto No.5 de MT404, verifica-se que elas são semelhantes.
\begin{table}[!htb]
    \centering
    \caption{Informações obtidas nos testes computacionais.}
    \label{tab:res}
    \csvautotabular{src/mt404_p06.csv}
\end{table}

Alguns pontos que chamam a atenção na Tabela~\ref{tab:res} são:
\begin{enumerate}
    \item O resíduo relativo no teste 1 é maior que os demais;
    \item O erro relativo nos testes 5 e 6 foram iguais a 1.
\end{enumerate}
A explicar essas observações ficam para trabalhos futuros.

\section{C\'{o}digos}
A seguir encontra-se os c\'{o}digos desenvolvidos neste projeto e que
encontram-se disponíveis em \url{https://github.com/r-gaia-cs/mt404-2012s2}.

Todos os c\'{o}digos foram implementados em Fortran e as instruções para
compilação encontram-se no Código~\ref{code:Makefile}. Para os testes
computacionais utilizou-se a versão do gfortran baseada no gcc 4.6.3.
\lstinputlisting[style=codes, caption={Rotina de teste},
label={code:mt404_p06}]{src/mt404_p06.f}
\lstinputlisting[style=codes, caption={Subrotinas envolvendo vetores},
label={code:vector}]{src/vector.f}
\lstinputlisting[style=codes, caption={Subrotinas envolvendo matrizes},
label={code:square_matrix}]{src/square_matrix.f}
\lstinputlisting[style=codes, caption={Subrotinas envolvendo a Decomposição de
Cholesky},
label={code:cholesky}]{src/cholesky.f}
\lstinputlisting[style=codes, caption={Makefile},
label={code:Makefile}]{src/Makefile}

\addcontentsline{toc}{section}{Refer\^{e}ncia Bibliogr\'{a}fica}
\bibliographystyle{alpha}
\bibliography{../referencias}
\end{document}
